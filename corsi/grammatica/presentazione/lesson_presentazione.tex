\mode
<article>

Questa è la presentazione del corso {\it Shell scripting e automazione
in ambienti Linux/Unix}.  Lo scopo dell'intervento è presentare
l'argomento del corso, le esigenze a cui risponde, gli obbiettivi e i
destinatari e la modalità di svolgimento e fruizione.


\mode
<all>

\mysection{L'argomento del corso}


\newcounter{gridvert}
\newcounter{gridoriz}
\newcounter{gridvertlen}
\newcounter{gridorizlen}

\newcommand\slidegrid[4]{
    \color{#1}
    \setcounter{gridoriz}{#3+1}
    \setcounter{gridvert}{#4+1}
    \setcounter{gridorizlen}{#3*#2}
    \setcounter{gridvertlen}{#4*#2}
    \multiput(0,0)(#2,0){\value{gridoriz}}{\line(0,1){\value{gridvertlen}}}
    \multiput(0,0)(0,#2){\value{gridvert}}{\line(1,0){\value{gridorizlen}}}
}

\newcommand\includeimg[3]{\resizebox{#1}{#2}{\includegraphics{#3}}}

\myframe{Cos'è la shell?}{frame:argomento}{
  \setlength{\unitlength}{1.5mm}

  \begin{picture}(60,20)(0,0)
    \color{black}
    \put(25,-4){\parbox{10\unitlength}{\centering hardware}}
    \put(25,1){\parbox{10\unitlength}{\centering kernel}}
      \qbezier(20,0)(20,5)(30,5)\qbezier(30,5)(40,5)(40,0)
      \put(20,0){\line(1,0){20}}

    \put(25,6){\parbox{10\unitlength}{\centering librerie}}
      \qbezier(15,0)(15,10)(30,10)\qbezier(30,10)(45,10)(45,0)
      \put(15,0){\line(1,0){30}}
    
    
    \color{engred}\put(25,11){\parbox{10\unitlength}{\centering\bf shell}}
      \qbezier(10,0)(10,15)(30,15)\qbezier(30,15)(50,15)(50,0)
      \put(0,0){\line(1,0){60}}
    

      \put(5,5){\includeimg{5\unitlength}{!}{clipart/people_juliane_krug_07c.png}}
      \put(8,9){\includeimg{5\unitlength}{!}{clipart/people_juliane_krug_07a.png}}
      \put(12,12){\includeimg{5\unitlength}{!}{clipart/people_juliane_krug_08a.png}}
      \put(16,14){\includeimg{5\unitlength}{!}{clipart/people_juliane_krug_08b.png}}

      \put(50,5){\includeimg{5\unitlength}{!}{clipart/exec.png}}
      \put(47,9){\includeimg{5\unitlength}{!}{clipart/exec.png}}
      \put(43,12){\includeimg{5\unitlength}{!}{clipart/exec.png}}
      \put(38,14){\includeimg{5\unitlength}{!}{clipart/exec.png}}

  \end{picture}
}

%% \myframe{Cos'è la shell?}{frame:argomento}{
%%   \setlength{\unitlength}{1.5mm}

%%   \begin{picture}(60,30)(0,0)
%%     %% \slidegrid{cyan}{1}{60}{40}
%%     %% \slidegrid{magenta}{5}{12}{8}
    
%%     \color{black}

%%     \put(5,16){\recttesto{50}{8}{shell}}
%%     \put(5,0){\recttesto{22}{8}{interfaccia verso il sistema operativo}}
%%     \put(33,0){\recttesto{22}{8}{linguaggio di programmazione}}

%%     \put(12,12){\wiconbox{6mm}{clipart/download_manager.png}}
%%     \put(40,12){\wiconbox{6mm}{clipart/download_manager.png}}

%%   \end{picture}
%% }

\mysection{Le esigenze a cui il corso risponde}



\myframe{Come è utilizzata la shell attualmente?}{frame:comeusata}{
  \vfill
\begin{center}
  \begin{tikzpicture}[x=2mm,y=2mm,node distance=1,
      box/.style={rounded corners,
        minimum height=1.2cm,
        minimum width=2.7cm,
        align=center,
        top color=#1!20,
        text width=2.7cm,
        bottom color=#1!50,},
      shell/.style={box=red},
      action/.style={box=blue},
      evid/.style={rounded corners,
        top color=red!20!yellow,
        bottom color=red},
    ]

    \node[action] (att sist) {attività sistemistiche};
    \node[action] (avvio) [left=of att sist] {avvio di programmi};
    \visible<2->{
      \node[evid] (att sist evid) [below=of att sist,text width=5mm] {};
      \node[evid] (avvio evid) [below=of avvio,text width=1mm] {};
    }

    \node[action] (cron) [below=of att sist evid] {schedulazioni};
    \node[action] (batch) [below=of avvio evid] {esecuzione di batch};

    \visible<2->{
      \node[evid] (cron evid) [below=of cron,text width=9mm] {};
      \node[evid] (batch evid) [below=of batch,text width=25mm] {};
    }
  \end{tikzpicture}
\end{center}
\vfill

}


  %% \begin{picture}(60,40)(0,0)
  %%   \slidegrid{cyan}{1}{60}{40}
  %%   \slidegrid{magenta}{5}{12}{8}
    
  %%   \color{engred}

  %%   \put(10,4){\circle*{2}}
  %%   \put(50,4){\circle*{1}}
  %%   \put(10,28){\circle*{3}}
  %%   \put(50,28){\circle*{10}}

  %%   \color{black}

  %%   \put(20,12){\recttesto{20}{8}{shell}}
  %%   \put(0,0){\recttesto{20}{8}{attività sistemistiche}}
  %%   \put(40,0){\recttesto{20}{8}{avvio di programmi}}
  %%   \put(0,24){\recttesto{20}{8}{schedulazioni}}
  %%   \put(40,24){\recttesto{20}{8}{esecuzione batch}}


  %% \end{picture}



\myframe{Come potrebbe essere utilizzata la shell?}{frame:comenonusata}{
  \vfill
\begin{center}
  \begin{tikzpicture}[x=2mm,y=2mm,node distance=1,
      box/.style={rounded corners,
        minimum height=3cm,
        minimum width=2.7cm,
        align=center,
        top color=#1!20,
        text width=2.7cm,
        bottom color=#1!50,},
      shell/.style={box=red},
      action/.style={box=blue},
      evid/.style={rounded corners,
        top color=red!20!yellow,
        bottom color=red},
    ]

    \node[action] (auto) {automazione di processi};
    \node[action] (elab) [left=of auto] {elaborazione di dati};
  \end{tikzpicture}
\end{center}
\vfill

}

\mysection{Destinatari}

\myframe{Chi dovrebbe seguire questo corso?}{frame:destinatari}{
  \vfill

\definecolor{colora}{RGB}{150,255,150}
\definecolor{colorb}{RGB}{180,255,180}

\begin{center}
  \begin{tikzpicture}[x=2mm,y=2mm,node distance=1,
      box/.style={rounded corners,
        minimum height=2cm,
        minimum width=2.7cm,
        align=center,
        %text=#1!50!black,
        top color=#1!20,
        text width=2.7cm,
        bottom color=#1!50,},
      ok/.style={box=green},
      forse/.style={box=colora},
      dovrebbe/.style={rounded corners,
        minimum height=2cm,
        minimum width=2.7cm,
        text width=5cm,
        align=center,
        text=black,
        top color=red!20!yellow,
        bottom color=red,},
      esempi/.style={rounded corners,
        minimum width=2.5cm,
        align=center,
        text=black},
    ]

    \node[ok] (admin) {amministratori \\unix/linux};
    \node[forse] (utenti) [right=of admin] {utenti \\unix/linux};
    \node[forse] (programmatori) [right=of utenti] {programmatori};

    \node (step) [below=of utenti] { };

    \only<2->{ \node[dovrebbe] (dovrebbe) [below=of step] {chi necessita di automatizzare processi ed elaborare file} };

    
  \end{tikzpicture}
\end{center}
\vfill


}

\myframe{Prerequisiti}{frame:prerequisiti}{
  \vfill

\definecolor{colora}{RGB}{150,255,150}
\definecolor{colorb}{RGB}{180,255,180}

\begin{center}
  \begin{tikzpicture}[x=2mm,y=2mm,node distance=1,
      box/.style={rounded corners,
        minimum height=2cm,
        minimum width=2.7cm,
        align=center,
        %text=#1!50!black,
        top color=#1!30,
        text width=2.7cm,
        bottom color=#1!80,},
      ok/.style={box=green},
      meglio/.style={box=yellow},
      obbligo/.style={box=red},
    ]

    \node[meglio] (base) {strutture base di un linguaggio procedurale};
    \node[ok] (familiarita) [right=of base] {familiarità con un ambiente Linux/Unix};
    \node[obbligo] (nessuno) [left=of base] { };

    \node (baseb) [below=of base] {\tiny\it fortemente consigliato};
    \node (famb) [below=of familiarita] {\tiny\it consigliato};
    \node (nessunob) [below=of nessuno] {\tiny\it obbligatorio};
    
  \end{tikzpicture}
\end{center}
\vfill


}

%\myframe{Casi d'uso delle conoscenze}{frame:casidestinatari}{}

\mysection{Obbiettivi didattici}

\myframe{Obbiettivi}{frame:obbiettivi}{
  \vfill

\definecolor{colora}{RGB}{150,255,150}
\definecolor{colorb}{RGB}{180,255,180}

\begin{center}
\begin{footnotesize}
  \begin{tikzpicture}[x=2mm,y=2mm,node distance=1,
      box/.style={rounded corners,
        minimum height=2cm,
        minimum width=1.5cm,
        align=center,
        %text=#1!50!black,
        top color=#1!30,
        text width=1.75cm,
        bottom color=#1!80,},
      teoria/.style={box=orange},
      pratica/.style={box=yellow},
      veri/.style={box=green},
      label/.style={rounded corners,
        minimum height=4mm,
        minimum width=3cm,
        align=center,
        text=#1!50!black,
        top color=#1!10,
        bottom color=#1!50,},
      labelt/.style={label=orange},     
      labelp/.style={label=yellow},     
    ]

    \node[pratica] (shell) {conoscere il funzionamento della shell};
    \node[pratica] (comandi) [right=of shell] {conoscere i comandi base};
    \node[teoria] (regexp) [right=of comandi] {conoscere le espressioni regolari};
    \node[teoria] (so) [right=of regexp] {approfondire aspetti di Linux/Unix};

    \node[labelp] (pratica) at ($(comandi.north)!.5!(shell.north)+(0,2)$) {\it pratica};
    \node[labelt] (teoria) at ($(regexp.north)!.5!(so.north)+(0,2)$) {\it teoria};

    \visible<2->{
      \node (arrow) at ($(comandi.south)!.5!(regexp.south)+(0,-3)$) {\wiconbox{6mm}{clipart/download_manager.png}};
      
      \node[veri] (file) [below=of arrow,text width=2cm] {imparare a elaborare file};
      \node[veri] (mant) [left=of file,text width=2cm] {imparare a scrivere shell script efficienti e mantenibili};
      \node[veri] (auto) [right=of file,text width=2cm] {imparare ad \\automatizzare task};
    }
  \end{tikzpicture}
\end{footnotesize}
\end{center}
\vfill


}

\myframe{Conoscenze in uscita}{frame:conoscenze}{
  \vfill

\definecolor{colora}{RGB}{150,255,150}
\definecolor{colorb}{RGB}{180,255,180}

\begin{center}
\begin{footnotesize}
  \begin{tikzpicture}[x=2mm,y=2mm,node distance=1,
      box/.style={rounded corners,
        minimum height=1.5cm,
        minimum width=1cm,
        align=center,
        %text=#1!50!black,
        text width=2cm,
        top color=#1!30,
        bottom color=#1!80,},
      teoria/.style={box=orange},
      pratica/.style={box=yellow},
      veri/.style={box=green},
      label/.style={rounded corners,
        align=center,
        text=#1!50!black,
        text width=2cm,
        top color=#1!10,
        bottom color=#1!50,},
      labelt/.style={label=orange},     
      labelp/.style={label=yellow},     
    ]

    \node[pratica] (usare) {usare la shell};
    \node[pratica] (usare cmd) [right=of usare] {usare i comandi base};
    \node[pratica] (combinare) [right=of usare cmd] {combinare i comandi};
    \node[pratica] (usare re) [right=of combinare] {usare espressioni regolari};


    \node[teoria] (shell) [above=of usare] {descrivere la shell};
    \node[teoria] (comandi) [above=of usare cmd] {descrivere i comandi base};
    \node[teoria] (regexp) [above=of usare re] {descrivere le espressioni regolari};
    \node[teoria] (concetti) [above=of combinare] {esporre concetti fondamentali};

    %% \draw[->] (concetti) to (shell);
    %% \draw[->] (concetti) to (comandi);
    %% \draw[->] (shell) to (usare);
    %% \draw[->] (shell) to (combinare);
    %% \draw[->] (shell) to (usare re);
    %% \draw[->] (comandi) to (usare cmd);
    %% \draw[->] (comandi) to (combinare);
    %% \draw[->] (comandi) to (usare re);
    %% \draw[->] (regexp) to (usare re);

    \node (arrow) at ($(usare cmd.south)!.5!(combinare.south)+(0,-3)$) {\wiconbox{6mm}{clipart/download_manager.png}};

    \node[veri] (elaborare) [below=of arrow] {elaborare file e dati};
    \node[veri] (efficienti) [left=of elaborare] {scrivere shell script efficienti e mantenibili};
    \node[veri] (automatizzare) [right=of elaborare] {automatizzare task};


  \end{tikzpicture}
\end{footnotesize}
\end{center}
\vfill


}


\mysection{Metodologia didattica}

\myframe{Programma didattico}{frame:programma}{
  \vfill

\definecolor{colora}{RGB}{150,255,150}
\definecolor{colorb}{RGB}{180,255,180}

\begin{center}
\begin{footnotesize}
  \begin{tikzpicture}[x=2mm,y=2mm,node distance=1,
      box/.style={rounded corners,
        minimum height=1.5cm,
        minimum width=1cm,
        align=center,
        %text=#1!50!black,
        text width=2cm,
        top color=#1!30,
        bottom color=#1!80,},
      teoria/.style={box=orange},
      pratica/.style={box=yellow},
      veri/.style={box=green},
      label/.style={rounded corners,
        align=center,
        text=#1!50!black,
        text width=2cm,
        top color=#1!10,
        bottom color=#1!50,},
      labelt/.style={label=orange},     
      labelp/.style={label=yellow},     
    ]

    \node[pratica] (linguaggio)  {la shell come linguaggio};
    \node[pratica] (orchestratore) [right=of linguaggio] {la shell come orchestratore};
    \node[pratica] (comandi) [right=of orchestratore] {comandi base};
    \node[pratica] (awksed) [right=of comandi] {awk\\sed};

    \node[teoria] (shell) at ($(linguaggio.north)!.5!(orchestratore.north)+(0,5)$)  {il ruolo della shell};
    \node[teoria] (concetti) [right=of shell] {concetti base di Linux/Unix};
    \node[teoria] (regexp) [right=of concetti] {espressioni regolari};


    \node[veri] (efficienti) at ($(linguaggio.south)!.5!(orchestratore.south)+(0,-5)$)  {best practice};
    \node[veri] (elaborare) [right=of efficienti] {casi di studio: elaborare file};
    \node[veri] (automatizzare) [right=of elaborare] {casi di studio: automatizzare};


  \end{tikzpicture}
\end{footnotesize}
\end{center}
\vfill


}


\myframe{Tipi di modulo}{frame:modulo}{
  \vfill

\definecolor{colora}{RGB}{150,255,150}
\definecolor{colorb}{RGB}{180,255,180}

\begin{center}
\begin{footnotesize}
  \begin{tikzpicture}[x=2mm,y=2mm,node distance=1,
      box/.style={rounded corners,
        minimum height=2.2cm,
        minimum width=1cm,
        align=center,
        %text=#1!50!black,
        text width=2cm,
        top color=#1!30,
        bottom color=#1!80,},
      teoria/.style={box=orange},
      pratica/.style={box=yellow},
      veri/.style={box=green},
      label/.style={rounded corners,
        align=center,
        text=#1!50!black,
        text width=2cm,
        top color=#1!10,
        bottom color=#1!50,},
      labelt/.style={label=orange},     
      labelp/.style={label=yellow},     
    ]

    \node[teoria] (teoria)  {modulo per concetti teorici\\(20-30 minuti)};
    \node[pratica] (pratica) [right=of teoria,text width=4cm]  {modulo per aspetti pratici\\(40-60 minuti)};
    \node[veri] (casi) at ($(teoria.south)!.66!(pratica.south)+(0,-7)$) [text width=6.4cm] {modulo in cui si affronta un caso di studio\\(60-120 minuti)};


  \end{tikzpicture}
\end{footnotesize}
\end{center}
\vfill


}

\myframe{Struttura di un modulo}{frame:struttura}{
\vfill
\begin{center}\begin{footnotesize}
\begin{tabular}{|p{2cm}|p{2cm}|p{2cm}|p{2cm}|}
\hline
& {\bf modulo di teoria}&{\bf modulo di pratica}&{\bf caso di studio}\\
& {\centering 20-30 min. }& {\centering 40-60 min. }& {\centering 60-120 min.} \\
\hline
{\it lezione frontale (10 min.)} & \multicolumn{2}{|c|}{lezione frontale} & esposizione del caso di studio\\
\hline
{\it esercitazione} & test a domande aperte, quiz, discussione, ...
& esercizi di programmazione, test a domande aperte, quiz, lavori di gruppo, ...
& progetto individuale o di gruppo sul caso di studio presentato\\
\hline
\end{tabular}
\end{footnotesize}\end{center}
\vfill
}

%% \myframe{Online o in presenza?}{frame:online}{
%% I moduli possono essere usufruiti sia in aula che a distanza (o in combinazione). 

%% Nel caso di  lezione in aula, le esercitazioni  potranno essere svolte
%% in   modalità   individuale,  a   gruppi   o   collegiale  a   seconda
%% dell'argomento e  dell'attitudine della classe. 

%% Nel  caso di lezione  a distanza,  la lezione  frontale sarà  un breve
%% video,  seguita  da   esercitazioni  svolte  individualmente.   

%% Queste  esercitazioni   potranno  essere  seguite   dall'insegnante  o
%% valutate in  modalità peer-reviewed, oppure  prevedere una discussione
%% tramite  chat o  forum, a  seconda dell'argomento,  della  modalità di
%% fruizione e svolgimento del corso e dell'attitudine dei partecipanti.

%% Nel caso di corso in aula, ogni giornata tranne l'ultima prevederà uno
%% o  due  moduli  teorici, cinque  o  sei  moduli  pratici e  un  modulo
%% riepilogativo  finale. L'ultima  giornata  avrà tre  o quattro  moduli
%% pratici e due moduli riepilogativi.

%% Nel  caso di  corso a  distanza,  il partecipante  potrà scegliere  il
%% proprio percorso, purché rispetti eventuali prerequisiti tra i moduli.
%% }

