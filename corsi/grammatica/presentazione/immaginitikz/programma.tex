\vfill

\definecolor{colora}{RGB}{150,255,150}
\definecolor{colorb}{RGB}{180,255,180}

\begin{center}
\begin{footnotesize}
  \begin{tikzpicture}[x=2mm,y=2mm,node distance=1,
      box/.style={rounded corners,
        minimum height=1.5cm,
        minimum width=1cm,
        align=center,
        %text=#1!50!black,
        text width=2cm,
        top color=#1!30,
        bottom color=#1!80,},
      teoria/.style={box=orange},
      pratica/.style={box=yellow},
      veri/.style={box=green},
      label/.style={rounded corners,
        align=center,
        text=#1!50!black,
        text width=2cm,
        top color=#1!10,
        bottom color=#1!50,},
      labelt/.style={label=orange},     
      labelp/.style={label=yellow},     
    ]

    \node[pratica] (linguaggio)  {la shell come linguaggio};
    \node[pratica] (orchestratore) [right=of linguaggio] {la shell come orchestratore};
    \node[pratica] (comandi) [right=of orchestratore] {comandi base};
    \node[pratica] (awksed) [right=of comandi] {awk\\sed};

    \node[teoria] (shell) at ($(linguaggio.north)!.5!(orchestratore.north)+(0,5)$)  {il ruolo della shell};
    \node[teoria] (concetti) [right=of shell] {concetti base di Linux/Unix};
    \node[teoria] (regexp) [right=of concetti] {espressioni regolari};


    \node[veri] (efficienti) at ($(linguaggio.south)!.5!(orchestratore.south)+(0,-5)$)  {best practice};
    \node[veri] (elaborare) [right=of efficienti] {casi di studio: elaborare file};
    \node[veri] (automatizzare) [right=of elaborare] {casi di studio: automatizzare};


  \end{tikzpicture}
\end{footnotesize}
\end{center}
\vfill

