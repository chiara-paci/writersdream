\vfill

\definecolor{colora}{RGB}{150,255,150}
\definecolor{colorb}{RGB}{180,255,180}

\begin{center}
\begin{footnotesize}
  \begin{tikzpicture}[x=2mm,y=2mm,node distance=1,
      box/.style={rounded corners,
        minimum height=1.5cm,
        minimum width=1cm,
        align=center,
        %text=#1!50!black,
        text width=2cm,
        top color=#1!30,
        bottom color=#1!80,},
      teoria/.style={box=orange},
      pratica/.style={box=yellow},
      veri/.style={box=green},
      label/.style={rounded corners,
        align=center,
        text=#1!50!black,
        text width=2cm,
        top color=#1!10,
        bottom color=#1!50,},
      labelt/.style={label=orange},     
      labelp/.style={label=yellow},     
    ]

    \node[pratica] (usare) {usare la shell};
    \node[pratica] (usare cmd) [right=of usare] {usare i comandi base};
    \node[pratica] (combinare) [right=of usare cmd] {combinare i comandi};
    \node[pratica] (usare re) [right=of combinare] {usare espressioni regolari};


    \node[teoria] (shell) [above=of usare] {descrivere la shell};
    \node[teoria] (comandi) [above=of usare cmd] {descrivere i comandi base};
    \node[teoria] (regexp) [above=of usare re] {descrivere le espressioni regolari};
    \node[teoria] (concetti) [above=of combinare] {esporre concetti fondamentali};

    %% \draw[->] (concetti) to (shell);
    %% \draw[->] (concetti) to (comandi);
    %% \draw[->] (shell) to (usare);
    %% \draw[->] (shell) to (combinare);
    %% \draw[->] (shell) to (usare re);
    %% \draw[->] (comandi) to (usare cmd);
    %% \draw[->] (comandi) to (combinare);
    %% \draw[->] (comandi) to (usare re);
    %% \draw[->] (regexp) to (usare re);

    \node (arrow) at ($(usare cmd.south)!.5!(combinare.south)+(0,-3)$) {\wiconbox{6mm}{clipart/download_manager.png}};

    \node[veri] (elaborare) [below=of arrow] {elaborare file e dati};
    \node[veri] (efficienti) [left=of elaborare] {scrivere shell script efficienti e mantenibili};
    \node[veri] (automatizzare) [right=of elaborare] {automatizzare task};


  \end{tikzpicture}
\end{footnotesize}
\end{center}
\vfill

