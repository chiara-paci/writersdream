\vfill

\definecolor{colora}{RGB}{150,255,150}
\definecolor{colorb}{RGB}{180,255,180}

\begin{center}
\begin{footnotesize}
  \begin{tikzpicture}[x=2mm,y=2mm,node distance=1,
      box/.style={rounded corners,
        minimum height=2cm,
        minimum width=1.5cm,
        align=center,
        %text=#1!50!black,
        top color=#1!30,
        text width=1.75cm,
        bottom color=#1!80,},
      teoria/.style={box=orange},
      pratica/.style={box=yellow},
      veri/.style={box=green},
      label/.style={rounded corners,
        minimum height=4mm,
        minimum width=3cm,
        align=center,
        text=#1!50!black,
        top color=#1!10,
        bottom color=#1!50,},
      labelt/.style={label=orange},     
      labelp/.style={label=yellow},     
    ]

    \node[pratica] (shell) {conoscere il funzionamento della shell};
    \node[pratica] (comandi) [right=of shell] {conoscere i comandi base};
    \node[teoria] (regexp) [right=of comandi] {conoscere le espressioni regolari};
    \node[teoria] (so) [right=of regexp] {approfondire aspetti di Linux/Unix};

    \node[labelp] (pratica) at ($(comandi.north)!.5!(shell.north)+(0,2)$) {\it pratica};
    \node[labelt] (teoria) at ($(regexp.north)!.5!(so.north)+(0,2)$) {\it teoria};

    \visible<2->{
      \node (arrow) at ($(comandi.south)!.5!(regexp.south)+(0,-3)$) {\wiconbox{6mm}{clipart/download_manager.png}};
      
      \node[veri] (file) [below=of arrow,text width=2cm] {imparare a elaborare file};
      \node[veri] (mant) [left=of file,text width=2cm] {imparare a scrivere shell script efficienti e mantenibili};
      \node[veri] (auto) [right=of file,text width=2cm] {imparare ad \\automatizzare task};
    }
  \end{tikzpicture}
\end{footnotesize}
\end{center}
\vfill

