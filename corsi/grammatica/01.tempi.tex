\usepackage[italian]{babel}

\mode
<presentation>
{
  \usetheme{WD}
  \setbeamercovered{transparent}
}

\mode
<all>

\mytitle{Tempi verbali}{I verbi nella narrazione}{La determinazione del tempo}

\myauthor{Chiara Paci}

\myinstitute{Writer's Dream}

\def\myplace{http://www.writersdream.org}

\date{\today}
%\date{21 gennaio 2015}

\setcounter{tocdepth}{1}

\pgfdeclareimage[height=0.8cm]{plogo}{wdlogo}
\logo{\pgfuseimage{plogo}}

% \def\banner{banner-zhang.png}
\def\banner{banner-skylitzes.png}

% Delete this, if you do not want the table of contents to pop up at
% the beginning of each subsection:
%% \AtBeginSection[]{
%%   \begin{frame}<beamer>
%%     \frametitle{Indice}
%%     \tableofcontents[sectionstyle=show/shaded,subsectionstyle=hide/hide/hide]
%%   \end{frame}
%% }

\mode
<all>

\newcounter{gridvert}
\newcounter{gridoriz}
\newcounter{gridvertlen}
\newcounter{gridorizlen}

\newcommand\slidegrid[4]{
    \color{#1}
    \setcounter{gridoriz}{#3+1}
    \setcounter{gridvert}{#4+1}
    \setcounter{gridorizlen}{#3*#2}
    \setcounter{gridvertlen}{#4*#2}
    \multiput(0,0)(#2,0){\value{gridoriz}}{\line(0,1){\value{gridvertlen}}}
    \multiput(0,0)(0,#2){\value{gridvert}}{\line(1,0){\value{gridorizlen}}}
}

\newcommand\includeimg[3]{\resizebox{#1}{#2}{\includegraphics{#3}}}

\begin{document}

\openpresentazione

\mode
<all>

\mode
<article>

In questa  lezione ci occupiamo  di verbi,  in particolare di  come si
coordinano tra  loro i  tempi verbali a  seconda che  vogliamo narrare
eventi  contemporanei, anteriori  o  posteriori all'evento  principale
della scena.

\mode
<all>

\mysection{Il tempo della narrazione}

\pgfdeclareradialshading{red}{\pgfpoint{0}{0}}%
                        {rgb(0bp)=(1,0.5,0);
                          rgb(20bp)=(1,0,0);
                          rgb(50bp)=(0.7,0,0);
                          rgb(55bp)=(1,1,1)}
\pgfdeclareradialshading{green}{\pgfpoint{0}{0}}%
                        {rgb(0bp)=(0.75,1,0.7);
                          rgb(20bp)=(0,1,0);
                          rgb(50bp)=(0,0.7,0);
                          rgb(55bp)=(1,1,1)}
\pgfdeclareradialshading{blue}{\pgfpoint{0}{0}}%
                        {rgb(0bp)=(0.5,0.75,1);
                          rgb(20bp)=(0,0,1);
                          rgb(50bp)=(0,0,0.7);
                          rgb(55bp)=(1,1,1)}
\pgfdeclareradialshading{yellow}{\pgfpoint{0}{0}}%
                        {rgb(0bp)=(1,1,1);
                          rgb(20bp)=(1,1,0);
                          rgb(50bp)=(1,.75,0);
                          rgb(55bp)=(1,1,1)}
                        %\pgfuseshading{red}


\begin{tikzfadingfrompicture}[name=fade out 2]
\shade[shading=radial,rotate=30,
  inner  color=transparent!0,
  outer color=transparent!100] % (0,0) circle[x radius=.3,y radius=.1];
           (5:1.2)--(5:2.2) 
           arc[radius=2.2,start angle=5,delta angle=110] 
           -- (5+110:1.2)
           arc[radius=1.2,start angle=5+110,delta angle=-110] ;
\end{tikzfadingfrompicture}

\myframe{La scena}{frame:scena}{
  \setlength{\unitlength}{1.5mm}
  \begin{center}
  \begin{tikzpicture}[x=1cm,y=1cm]

    \tikzfading[name=fade out,
      inner color=transparent!0,
      outer color=transparent!100]

    \foreach \x/\label/\cs/\op in {30/personaggi/yellow/0,150/luogo/green/0,270/tempo/blue/80} {
      \begin{scope}[rotate=\x]
      \begin{scope}[shading=\cs]
      \path[shade,rounded corners]
      %\path[draw,rounded corners]
           [preaction={transform canvas={scale around={1.1:(60+\x:1.5)}},
                       fill=white!50!black,fit fading=false,
                       fading transform={shift=(60+\x:1.45),scale=1,rotate=\x-30},
                       path fading={fade out 2}}] 
           [preaction={transform canvas={scale=1.06},
                       fill=white!50!black,fit fading=false,
                       fading transform={shift=(60+\x:1.45),scale=1,rotate=\x-30},
                       path fading={fade out 2}}] 
           (5:1.2)--(5:2.2) 
           arc[radius=2.2,start angle=5,delta angle=110] 
           -- (115:1.2)
           arc[radius=1.2,start angle=5+110,delta angle=-110] 
           ;
      \end{scope}

      %\path[draw] (60:1.7) circle [x radius=2,y radius=.5,rotate=-30];

      \path[decorate,decoration={text along path, 
                                 text align={fit to path},
                                 text={\label}}] 
           (+115-20:1.6) arc[radius=1.6,start angle=+115-20,delta angle=-70] ;

      \end{scope}
    }

    \begin{scope}[shading=red]
      \path[shade,blur shadow={shadow scale=1.05,shadow xshift=0,shadow yshift=0,shadow opacity=30}] (0,0) circle [radius=1];
    \end{scope}[shading=red]
    \draw (0,0) node (azione) {azione};

  \end{tikzpicture}
  \end{center}
}

\myframe{I momenti}{frame:momenti}{
  \begin{itemize}
    \item momento di riferimento;
    \item momento dell'enunciazione;
    \item momento dell'azione.
  \end{itemize}
}

\myframe{La localizzazione temporale}{frame:localizzazione}{}

\mysection{I verbi e la narrazione}

\myframe{Il tempo narrativo}{frame:temponarrativo}{}

\myframe{Il sistema verbale italiano}{frame:verbo}{}

\myframe{Perfetto e imperfetto}{frame:aspetto}{}

\myframe{Presente assoluto}{frame:assoluto}{}

\mysection{Schema del passato remoto}

\myframe{Schema complessivo}{frame:passatocomplessivo}

\myframe{Contemporaneità}{frame:passatocontemporaneo}

\myframe{Anteriorità}{frame:passatoanteriore}

\myframe{Posteriorità}{frame:passatoposteriore}

\mysection{Schema del presente}

\myframe{Schema complessivo}{frame:presentecomplessivo}

\myframe{Contemporaneità}{frame:presentecontemporaneo}

\myframe{Anteriorità}{frame:presenteanteriore}

\myframe{Posteriorità}{frame:presenteposteriore}



\end{document}




